\section{Electromagnetism of periodic structures} %{{{
The peculiar behaviour of electromagnetic waves in periodic structures has attracted human attention for ages, long before anybody perceived that light is an \textit{electromagnetic wave} or that it is the \textit{periodicity} is responsible for the brilliant and irreproducible colours of opal gemstones and many living creatures, such as various beetles, butterflies or peacocks. 

The scientific community started to study the underlying phenomena in the late 19th century when the X-ray scattering was observed on crystal lattices (W. H. and W. L. Bragg, \todo{19xx}) and when the high optical reflection from periodic layers of entirely transparent materials was predicted (Lord Rayleigh, 1870s). Note that not all related research was focused on electromagnetism -- for example, the Bloch theorem, discussed later in this thesis, was originally developed for the electron waves in solids (F. Bloch, \todo{19xx}).

Simultaneous theoretical and technological progress through the 20th century allowed to design custom structures that interact with electromagnetic waves in a desired way. In analogy with the rapidly growing field of \textit{electronics}, the term \textit{photonics} was coined \todo{in the 19xxs} for the design of optical fibers and other waveguides, sources, filters, detectors etc. The rapid development in this field was enabled by the modern technology of microfabrication, along with the unprecedented power of computers used for numerical simulations. Nowadays, photonics-based devices play a practically irreplaceable role in the high-bandwidth telecommunication, science and industry.

Out of all structures studied, arguably the largest theoretical attention was attracted to structures that are periodic.
Electromagnetic phenomena in periodic structures were studied in analogy with the physical theories established previously. One paradigm of \textit{photonic crystals} was inspired by the electron waves in crystals; slightly later the paradigm of \textit{metamaterials} was inspired by the electromagnetic waves in homogeneous materials. 

The key concept is that the electromagnetic properties of a periodic structure can be, at least partially, understood as those of a homogeneous medium, but in fact they are determined predominantly by the spatial arrangement of the structure unit cell. The actual constituent materials can be relatively freely chosen. % This allows one to take into account the technological and economical aspects. 
While it is unlikely that a radically new homogeneous material will be invented for construction of optical elements, periodic structuring of ordinary materials provides unprecedented freedom in tuning the electromagnetic properties and even enables to obtain phenomena unusual in homogeneous materials. 

\section{Motivation for terahertz photonics}
This work focuses on periodic structures operating in the terahertz (THz) spectral range, which spans roughly from 100 GHz to 10 THz. While the electrodynamic theory presented in this work is scale-invariant and can be used in the whole electromagnetic spectrum, the selected frequency range defines the properties of constituent materials and technological processes available. Compared to the well established optical technology (400---700 THz), the range of materials suitable for THz frequencies gives additional possibilities, such as the use of superconductors, extremely high-permittivity dielectrics and tunable ferroelectrics. Additionally, the much longer wavelength of terahertz waves, e.g. 300 $\upmu$m for 1 THz in free space, also enables much easier fabrication of the structures. On the other hand, materials such as glass, most plastics etc. commonly used in other spectral ranges must be avoided, since they exhibit excessively high losses in the terahertz range.

The terahertz range is located in the electromagnetic spectrum at the boundary between the microwave region where most often the paradigm of high-frequency electronics is used, and the infrared and optical region where classical optics is used \cite{ozyuzer2007emission}. At THz frequencies, concepts from both paradigms are often seamlessly used together: waves emitted from lumped semiconductor components can be collimated by lenses, picosecond pulses generated in nonlinear crystals can be detected by photoconductive sampling, etc. 
%% ... gap in the generation of electromagnetic radiation, extending ap- proximately from 0.5 THz to 2 THz, stems from the separation of the two general
%%paradigms for generating electromagnetic waves (1–3): alternating currents in semiconductor- based electronics and electronic transitions be- tween quantized electronic states in lasers,
%%respectively. The frequency of semiconductor devices is bounded from above by limits of the electron velocities, whereas the frequency of solid-
%%state lasers is bounded from below by thermal energies that limit the smallest electronic transitions useful for lasing...

Yet none of these approaches is optimal for the THz applications; from the electronic point of view, there is still demand for fast-enough diodes and power transistors and the microstrip circuits become too lossy at high frequencies. The optical approach is often burdened by the strong wave-optics phenomena such as diffraction, and moreover some light-matter interactions are weaker than at the optical frequencies, limiting the possibilities for e.g. amplitude modulation by Pockels effect. 

These deficiencies provide additional reasons to search for the new possibilities opened by photonic devices and periodic structures operating in the terahertz range.  

%}}}
%% \section{Aims of the thesis} ??
\section{Outline of the thesis} %{{{
\paragraph{Theory} %{{{
The theoretical chapter starts with a review of linear \textit{electrodynamics of continuous media} with an account for anisotropy. In its customary form without account for spatial dispersion, this topic is treated in every related textbook. Many related topics such as nonlinearity, optical activity and gyrotropy are not discussed here, as they are not essential for the description of metamaterials in the rest of the thesis.  

The classical electrodynamics for local media serves as a basis for the so-called Landau-Lifshitz (or, $EDB$) formulation of electrodynamics that can be easier extended also for spatially-dispersive media, that is, to account for the material properties depending not only on the frequency, but also on the magnitude of the wave vector.
%, which is elaborated more in detail.  ---  in fact I should write more there.
Multiple examples of weak spatial dispersion can be found in natural media, but for the description of  
In the author's opinion, this topic does not receive due attention in most of the literature.
%% ((( first we build the theory from first principles, and using it we then easier classify the structures)))

The Bloch theorem follows along with its mathematical proof, which enables to adapt the continuum electrodynamics also for waves propagating in periodic structures. 
The distinction of two widely recognized types of these, namely \textit{metamaterials} and \textit{photonic crystals}, is discussed from different aspects, with an emphasis on that the theoretical approaches used for each field can be unified and used for the other field as well.

%}}}
%%% TODO TODO TODO TODO TODO TODO 
%%% Hole 1: Outline of the thesis, write
%%% TODO TODO TODO TODO TODO TODO 
\paragraph{Experimental methods} %{{{
%The fourth section gives an overview of the experimental techniques that were used to fabricate the samples and measure their response to a broadband terahertz impulse.

% TODO Where appropriate, we also investigated how this behaviour depends on some external parameter (such as electric field, temperature or illumination) -- in other words, how the \textit{tunability} of the structure can be achieved.
% TODO is this true? Uncomment or remove! ...   Attention  was also paid also to investigate the tunability of their properties depending on external parameters, such as temperature, electric and magnetic fields and illumination. %, i.e. to 


%}}}
\paragraph{Numerical methods} %{{{
The  section focuses on the numerical methods we employed to predict experimental results and, most importantly, to understand the physical nature of the predicted phenomena. We provide a comparison of the finite-difference time-domain simulation (FDTD), the plane-wave expansion (PWE) and the transfer matrix method (TMM). We point out the capabilities of each of them and we also show how the results from these different methods can be processed to give comparable quantities.

%The dissertation involved the development of a reliable platform for numerical simulations of photonic structures, based on freely available code. 
%}}}
\paragraph{Results and conclusion} %{{{
%The longest section follows, in which a systematic list of the most important periodic structures is provided along with their electromagnetic behaviour. Through this work we focused on the terahertz spectral range. 

The results from the simulations were verified against experimental data and analytic models.

One of the aims of the dissertation is to give a comparison of these structures and to point out the profound similarities in their operation, which may not be obvious.  Some of the structures discussed were also manufactured and experimentally characterized during this PhD project. 

General conclusions and prospects are drawn in a separate section that follows the former.
%}}}
\paragraph{Appendices} %{{{
At the end of the thesis, several sections are devoted to topics which did not fit into the flow of the text. 

%}}}



%}}}
\section{Conventions used}%{{{
Thorough the thesis, we use a single or double apostrophe ($x', x''$) to refer to the real and imaginary part of a complex number ($x$). Italic symbols (e.g. $k$) represent the magnitudes of vectors, which are denoted by respective bold symbols (e.g. $k = |\kk|$). Components of vectors are denoted by small indices, such as $k_x, k_y, k_z$.

The explicit time, space or frequency dependence of quantities are omitted when it does not allow confusion. So for electric field we write $\E$ instead of $\E(\rr,t)$.

An important note shall be made on the sign convention for the complex wave, introduced in Eq. \ref{eq_pw}.
We use the `engineering' convention of time dependence: $e^{+\ii \omega t}$, but this is only due to the author's feeling that it is more natural when the wave phase grows in time. 
It is used in  roughly a half of the literature (e.g. \cite[p. 9]{engheta2006book}, \cite[pp. 21, 99]{krowne2007book}, \cite[(Chapters 1-4, 6, 9, 10)]{eleftheriades2005book}).  In the remaining part, (e.g. \cite[(Chapters 5, 7, 8)]{eleftheriades2005book}), \cite{klingshirn2007semiconductor}, \cite{jackson1962book}, \cite{veselago1968}, \cite{born1999book}, \cite[p. 5]{noginov2011book}), the opposite, `optical', convention is used with time dependence of $e^{-\ii \omega t}$. The choice of $e^{+\ii\omega t}$ or $e^{-\ii\omega t}$ determines the sign of the imaginary part in virtually all complex quantities discussed in this thesis, but with correct interpretation it makes no difference in the physical conclusions as it is only a formal simplification.
In the real world, observable fields do not have any imaginary component so the real part of the result has to be taken. 
%% ---- "ENGINEERING" in favor of +iwt , and then perhaps using eps = (eps' - i eps''), or getting along with all-negative eps''
%% https://www.comsol.com/support/knowledgebase/1009/    
%% http://www.tpdsci.com/tpc/RISignDv.php
%% Panofsky & Phillips 1962, p 200 ??
%% ---- "OPTICS" in favor of -iwt, then using the simpler eps = (eps' + i eps'')

In the $e^{+\ii\omega t}$ convention, many parameters of a passive (lossy) system are restricted to have \textit{negative imaginary part}.  An additional complication arises from that a part of the authors using $e^{+\ii\omega t}$ convention wish to represent the imaginary part as \textit{positive}, and they define complex quantities as, e.g., $\varepsilon = \varepsilon' - \ii \varepsilon''$, % TODO verify if this is e.g. "Wallen2011-Anti-resonant response of ..."
thus in their case $\varepsilon''\equiv -\text{Im}(\varepsilon)$. Thorough the thesis, we however represent the real and imaginary parts naturally as $\varepsilon := \varepsilon' + \ii \varepsilon''$.

The unit system differs thorough the literature, too. Some of the references, e.g. \cite{landau1984electrodynamics, agranovich2006spatial, krowne2007book_agran} use the older centimeter-gram-second (CGS) system, which for instance leaves out the dimension constants of $\varepsilon_0, \mu_0$. The whole thesis uses consistently the meter-kilogram-second (SI) system.
%}}}
