



\section{Introduction} % TODO
The behaviour of electromagnetic waves in periodic media has attracted the human attention for ages, long before anybody perceived that light is an \textit{electromagnetic wave} or that it is the \textit{periodicity} that is responsible for the brilliant and irreproducible colours of opal gemstones and many living creatures, such as various beetles, butterflies, peacocks etc. %{{{
The scientific community started to study the underlying phenomena in the late 19th century when the X-ray scattering was observed on (periodic) crystal lattices and also when the high optical reflection from periodic layers of entirely transparent materials was predicted. % TODO ref
%The technological and scientific boom of the 20th century contributed with many new theoretical approaches, numerical methods and types of periodic structures to the newly born field of \textit{photonics}. With the advent of the 21th century, great progress was also made in the research of \textit{metamaterials}, a specific subset of periodic structures which will be discussed in greater detail in this work. 
% TODO motivation - finish
The key concept in photonic crystal or metamaterial studies is that the electromagnetic properties are defined predominantly by the shape of the structure, while the actual materials that are used to build it can be relatively freely chosen. % This allows one to take into account the technological and economical aspects. 
While it is unlikely that a radically new material will be invented for construction of optical elements, several new phenomena can be obtained by periodic structuring of ordinary materials. The rapid development of this field was enabled by the modern technology of microfabrication, along with the unprecedented power of computers able to predict the structure behaviour.


% TODO introduction ??
% TODO THz science ??
This work focuses on the terahertz (THz) spectral range, which spans roughly from 100 GHz to 10 THz. While the electrodynamic theory presented in this work is scale-invariant and can be used from microwave to optical frequencies, the selected frequency range defined the properties of materials and technological processes available. Compared to the well established optical technology (400---700 THz), the range of materials suitable for THz frequencies gives additional possibilities, such as the use of superconductors, extremely high permittivity dielectrics and tunable ferroelectrics. Additionally, the much longer wavelength of terahertz waves, e.g. 300 $\upmu$m for 1 THz in free space, also enables much easier fabrication of the structures. On the other hand, some materials commonly found in the microwave or optical applications must be avoided, as they exhibit excessively high losses in the terahertz range (such as glass, water, most plastics etc.)

%\cite{lewis2014review} approximated the number of papers as an exponential, doubling every 3.2 years: D = 2**[(year-1975)/3.2]

The THz range is located in the spectrum at the boundary of the regions where people use "electronic" or "optical" approaches. At THz frequencies, devices from both paradigms are often seamlessly used together: waves from waveguides can be collimated by lenses, pulses emitted from lumped antenna emitters are detected by electrooptical crystals etc. Yet none of these approaches is optimal for the THz applications; from the electronic point of view, we are for instance still lacking transistors with fast enough response and the microstrip circuits become too lossy at high frequencies. The optical approach is often complicated by the strong wave-optics phenomena such as diffraction, while some light-matter interactions are weaker, limiting the possibilities for e.g. amplitude modulation by Pockels effect. These deficiencies give additional reasons to search for the new possibilities of the photonic crystals and metamaterials operating in the terahertz range.

% TODO THz MM review 


% TODO -> cíle studie a disertace
Many different designs of metamaterials  % and photonic crystals
were proposed in the last decades, part of them being aimed to the terahertz range. %They have been also summed by several books and reviews % TODO refs x5
% lacks:
%		not covering the area of possible structures
%		too much technology?
%		too specific, no comparison with similar structures, different frequency-spatial scalings of the same
%		often missing  cricital discussion of excessive losses, slow response etc.
%		missing proper electrodynamics
One of the aims of the dissertation are to give a comparison of these structures and to point out the profound similarities in their operation, which may not be obvious.  Some of the structures discussed were also manufactured and experimentally characterized during this PhD project. 
The second aim is to investigate the tunability of their properties depending on external parameters, such as temperature, electric and magnetic fields and illumination. %, i.e. to 
Last but not least, the dissertation project involves the development of a reliable platform for numerical simulations of photonic structures, based on freely available code. Thorough the thesis, the results from these simulations will be verified against experimental data and analytic models.
% Based on a firm theory of electrodynamics of periodic structures

%% TODO update 
%In the following section we present a brief overview of the relevant physical theory. We review the propagation of electromagnetic waves in the free space and in simple periodic structures. We  outline the boundary which usually divides the fields of \textit{photonic crystals} and \textit{metamaterials}, trying to support the hypothesis that the theoretical approaches used for each field can be unified and used for the other field as well.
%% ((( first we build the theory from first principles, and using it we then easier classify the structures)))
%The next section focuses on the numerical methods we employed to predict experimental results and, most importantly, to understand the physical nature of the predicted phenomena. We provide a comparison of the finite-difference time-domain simulation (FDTD), the plane-wave expansion (PWE) and the transfer matrix method (TMM). We point out the capabilities of each of them and we also show how the results from these different methods can be processed to give comparable quantities.
%The fourth section gives an overview of the experimental techniques that were used to fabricate the samples and measure their response to a broadband terahertz impulse.
%The longest section follows, in which a systematic list of the most important periodic structures is provided along with their electromagnetic behaviour. Through this work we focused on the terahertz spectral range. 
% TODO Where appropriate, we also investigated how this behaviour depends on some external parameter (such as electric field, temperature or illumination) -- in other words, how the \textit{tunability} of the structure can be achieved.
%In the last section, some general conclusions and prospects are drawn.

