% \newpage\vfill I declare I elaborated this thesis by myself. All literary sources and publications I have used had been cited.  \newpage 

\section{Electromagnetic waves in periodic structures} %{{{
The peculiar behaviour of light in periodic structures has attracted human attention for ages, long before anybody perceived that light is an \textit{electromagnetic wave} or that it is the \textit{periodicity} which is responsible for the brilliant and irreproducible colours of opal gemstones and many living creatures, such as various beetles, butterflies or peacocks. 

The scientific community started to rigorously study the underlying phenomena in the late 19th century.
Simultaneous theoretical and technological progress through the 20th century allowed to design custom structures that interact with electromagnetic waves in a desired way. In analogy with the rapidly growing field of \textit{electronics}, the term of \textit{photonics} was coined for the design of optical fibers and other waveguides, sources, filters, detectors etc. The rapid development in this field was enabled by the modern technology of microfabrication, along with the unprecedented power of computers used for numerical simulations. Nowadays, photonics-based devices play a vital role in the high-bandwidth telecommunication, science and industry.

%Electromagnetic phenomena in periodic structures were studied in analogy with the physical theories established previously % outside optics. 
Out of all structures studied in photonics, particularly detailed theoretical studies were devoted to structures that are periodic. Three paradigms related to the physics of periodic structures developed independently. While \textit{photonic crystals} were inspired by the electron waves, \textit{metamaterials} were inspired by the electromagnetic waves in crystals. Early treatises on electrodynamics of \textit{media with negative or unusual parameters} did not assume any structuring or other way of how these parameters should be achieved. On behalf of natural development, the paradigms unified into one, and this thesis tries to describe them in an unified manner.  % todo style

The essential concept is that the electromagnetic properties of a periodic structure can be, at least partially, understood as those of a homogeneous medium, but in fact they are determined predominantly by the spatial arrangement of the structure's unit cell. The actual constituent materials can be relatively freely chosen. While it is unlikely that a radically new homogeneous material will be invented for construction of optical elements, periodic structuring of ordinary materials provides unprecedented freedom in tuning the electromagnetic properties and even enables to obtain phenomena unusual in homogeneous materials.  % todo style

%}}}
\section{Motivation for terahertz photonics} %{{{
This work focuses on the electromagnetic behaviour of periodic structures operating in the terahertz (THz) spectral range, which spans roughly from 100 GHz to 10 THz. While the electrodynamic theory presented in this work is scale-invariant and can be used in the whole electromagnetic spectrum, the selected frequency range defines the properties of constituent materials and technological processes available. Compared to the well established optical technology (400---700 THz), the range of materials suitable for THz frequencies gives additional possibilities, such as the use of superconductors, extremely high-permittivity dielectrics and tunable ferroelectrics. Additionally, the much longer wavelength of terahertz waves, e.g. 300 $\upmu$m for 1 THz in free space, also enables much easier fabrication of relevant structures. On the other hand, materials such as glass and most plastics, commonly used in other spectral ranges, must be avoided, since they exhibit excessively high losses in the terahertz range.

The terahertz range is located in the electromagnetic spectrum at the boundary between the microwave region where most often the approaches of high-frequency electronics is used, and the infrared and optical region where classical optics is used \cite{ozyuzer2007emission}. At THz frequencies, both approaches are often seamlessly used together: waves emitted from lumped semiconductor components can be collimated by lenses, picosecond pulses generated in nonlinear crystals can be detected by photoconductive sampling, etc.  %% TODO - maybe write examples that are better covered by the use of MMs or PhCs

Yet none of these two approaches is optimal for the THz applications; from the electronic point of view, there is still demand for fast-enough semiconductor devices and also the microstrip circuits become too lossy at high frequencies. The optical approach is often burdened by the strong wave-optics phenomena such as diffraction, and moreover some light-matter interactions are weaker than at the optical frequencies, limiting the possibilities for e.g. amplitude modulation by the Pockels effect. 

These deficiencies provide additional reasons to search for the new possibilities opened by photonic devices and periodic structures operating in the terahertz range.  

%}}}

\section{Goals of the thesis}
\begin{enumerate}
\item{During the preparation of the thesis, its author encountered difficulties finding a focused, accessible, yet rigorous enough tutorial covering the necessary theory. The main aim of the theoretical section therefore lies in bridging the fundamental electrodynamics with the concepts used for description of metamaterials and photonic crystals.} 
\item{The importance of the more general \textit{spatially-dispersive} electrodynamics is emphasized, although it is unfortunately often neglected in the literature in favor of the simpler \textit{local} electrodynamics. Both approaches are compared throughout the thesis.} 
\item{The third aim of this thesis is to promote the unification of paradigms, i.e. \textit{photonic crystals}, \textit{metamaterials} and \textit{media of unusual parameters}, by showing that they can -- and should -- be described by the same theoretical approaches. This is supported by elaborating the concept of spatial dispersion in the theoretical section and by providing historical review of the surprisingly long parallel development of such structures.} 
\item{A overview of the electromagnetic behaviour of diverse periodic structures is provided in the Results section, taking into account also the properties of materials available for THz photonics, with the realistic level of absorption in particular. This section is more than just results -- proceeding from simpler structures to the more complex ones, it aims to explain all observed phenomena in a didactic manner. Some of the simulations are verified against the experimental results from the terahertz measurements.} 
\item{Last but not least, the numerical results present a small demonstration of the use of the extensive simulation scripts, which were developed solely for the thesis and are published online free of charge. The basic concepts of the numerical simulations are described to help others in adapting the simulations for their further research. } 
\end{enumerate}

\section{Thesis outline}
\paragraph{Theory} %{{{
The theoretical chapter starts with a review of linear \textit{electrodynamics of continuous media}; the concepts of harmonic oscillators, dispersion curves, and isofrequency contours are introduced. The generalized index of refraction is introduced, and it is demonstrated how the wave refraction depends on the shape of isofrequency contours.
%Many related topics, such as nonlinearity, optical activity and gyrotropy are not discussed here, as they are not essential for the selection of structures in rest of the thesis. 

The electrodynamic theory is generalized to account for the \textit{nonlocal} response, or equivalently \textit{spatial dispersion}, in the medium -- a phenomenon which can be usually neglected in the natural media, but is of key importance in any conscientious treatise on electrodynamics of periodic structures.
The more general Landau-Lifshitz formulation of nonlocal medium parameters is also introduced and shown to result in the same dispersion as the customary model.  

The third part of the theoretical section is dedicated to periodic structures, starting with the Bloch theorem that enables one to transfer some of the concepts from continuous optics even to waves propagating in periodic structures. Its impact on the form of dispersion curves and on the ambiguity of the wave vector is shown.

The distinction of two widely recognized types of periodic structures, namely \textit{metamaterials} and \textit{photonic crystals}, is discussed from historical
%, theoretical and practical 
aspects, with an emphasis on the author's view that the theoretical approaches used for each field can be unified and used for the other field as well.
% stress that the metamaterial and photonic crystal fields should be unified in terminology and approaches

%}}}
\paragraph{Experimental methods} %{{{
As a part of the research several experiments in the terahertz laboratory were performed. This section is opened by a systematic review of the terahertz sources and detectors, followed by the description of the terahertz time-domain spectroscopy setup used in our laboratory for characterisation of the samples. 

The fabrication and characterisation of the titanium dioxide microspheres samples is described in detail, with particular emphasis on the newly developed technique of acoustic-wave assisted sorting on anisotropic sieves. The section is concluded by the description of laser micromachining of the sieves and of the metallic meshes, which were used as electromagnetic filters.

%}}}
\paragraph{Numerical methods} %{{{
This section describes the tools and methods used to calculate the electromagnetic behaviour of periodic structures and, even more importantly, to understand the physical nature of the predicted phenomena. Its structure reflects the separation between the algorithms that solve the Maxwell equations as a purely numerical problem, and between their particular application.

Its first part provides a comparison of the finite-difference time-domain simulation (FDTD), its frequency-domain modification (FDFD) 
and the plane-wave expansion methods (PWEM).
%and the transfer matrix method (TMM). 
We point out the capabilities of each of them and we also briefly review multiple other methods that were not used here. % and we also show how the results from these different methods can be processed to give comparable quantities.
Defining numerically stable FDTD simulations with realistic materials, one encounters several pitfalls. They were painstakingly resolved during the preparation of the thesis, and are also presented here.

The second part describes the setups of the "numerical experiments" that are built atop the algorithms. 
The time-domain simulation is first used in conjunction with the \textit{s}-parameter retrieval setup, which is based on transmitting a broadband pulse towards a metamaterial unit cell and recovering its effective parameters from the reflected and transmitted fields. While this setup is shown to give mostly realistic results, however, it has its limitations which are also presented. Although the \textit{s}-parameter retrieval seems by far the most commonly used, different setups were described in the literature which are briefly noted.
One of such simulation setups of the current-driven homogenization is presented in details, and it is shown that its weaker prior assumptions on the structure behaviour lead to higher reliability.
% TODO one should really devise an algorithm for the Landau-Lifshitz permittivity retrieval from CDH (i.e. calibrate the source first on vacuum?)
%}}}
\paragraph{Results, conclusion and appendices} %{{{				%% TODO
%The perhaps most important section follows, 
%in which a systematic overview of the most important periodic structures is provided
The Results section puts the common metamaterial types into a perspective, pointing out what they have in common, and how they differ.  % TODO write which structures are discussed
While the space of all imaginable structures can never be covered, most typical classes  another aim of this section is to demonstrate the capabilities of the simulation scripts used. 
%for metamaterial simulation

The scale of the simulated structures is chosen for the structures to operate in the terahertz range. 
Thanks to the scale invariance of Maxwell equations, many of the designs and observations can be transposed also into other spectral regions, be it microwave or infrared. However, one has to bear in mind that the properties of the constituent materials may be different in other spectral regions. Naturally, also the techniques of fabrication and characterisation may be different.

Some of the structures discussed were also fabricated and experimentally characterized during this PhD project. 
The results from the simulations were verified against experimental data and analytic models.

At the end of the thesis, somewhat critical conclusions from the above results are drawn, along with directions in which the research could be pursued in future.
%% TODO about the appendices

%}}}

\section{Conventions used}%{{{
Throughout the thesis, a single or double apostrophe ($x', x''$) refers to the real and imaginary part of a complex number ($x$). Italic symbols represent the magnitudes of vectors, which are denoted by respective bold symbols (e.g. $k = |\kk|$). Components of vectors are denoted by small indices, such as $k_x, k_y, k_z$. 

The lists of symbols and abbreviations used are in Tables \ref{tb_symbols} and \ref{tb_shortcuts}, respectively. 
%The explicit time, space or frequency dependence of quantities are omitted when it can not cause confusion.

An important note shall be made on the sign convention for the complex wave, as introduced in Eq. (\ref{eq_pw}).
The `engineering' convention of time dependence of the complex exponential $e^{+\ii \omega t}$ is used, but this is only due to the author's feeling that it is more natural when the wave phase grows in time. Such a convention is shared with roughly a half of the literature, e.g. \cite[p. 9]{engheta2006book}, \cite[pp. 21 and 99]{krowne2007book}, \cite[Chapters 1-4, 6, 9 and 10]{eleftheriades2005book}.  In the remaining part, e.g. \cite[chapters 5, 7, 8]{eleftheriades2005book}, \cite{klingshirn2007semiconductor}, \cite{jackson1962book}, \cite{veselago1968}, \cite{born1999book}, \cite[p. 5]{noginov2011book}, the opposite, `optical', convention is used with time dependence of $e^{-\ii \omega t}$. The choice of $e^{+\ii\omega t}$ or $e^{-\ii\omega t}$ determines the sign of the imaginary part in virtually all complex quantities discussed in this thesis, but with correct interpretation it makes no difference in the physical conclusions.
%as it is only a formal simplification.
In the real world, observable fields do not have any imaginary component so the real part of the result has to be taken. 
%% ---- "ENGINEERING" in favor of +iwt , and then perhaps using eps = (eps' - i eps''), or getting along with all-negative eps''
%% https://www.comsol.com/support/knowledgebase/1009/    
%% http://www.tpdsci.com/tpc/RISignDv.php
%% Panofsky & Phillips 1962, p 200 ??
%% ---- "OPTICS" in favor of -iwt, then using the simpler eps = (eps' + i eps'')

In the $e^{+\ii\omega t}$ convention, many parameters of a passive (lossy) system are restricted to have \textit{negative imaginary part}. An additional complication arises from that a part of the authors using $e^{+\ii\omega t}$ convention still wish to represent the imaginary part as \textit{positive}, and they define complex quantities as, e.g., $\varepsilon = \varepsilon' - \ii \varepsilon''$, % TODO verify if this is e.g. "Wallen2011-Anti-resonant response of ..."
thus in their case $\varepsilon''\equiv -\text{Im}(\varepsilon)$. This is not the case of this thesis, and the real and imaginary parts are represented naturally as $\varepsilon := \varepsilon' + \ii \varepsilon''$.

The unit system differs across the literature, too. Some of the references, e.g. \cite{landau1984electrodynamics, agranovich2006spatial, krowne2007book_agran} use the older centimeter-gram-second (CGS) system, which for instance leaves out the dimension constants of $\varepsilon_0, \mu_0$. The whole thesis uses consistently the meter-kilogram-second (SI) system.

Having listed conventions we use, we should mention one convention we will avoid using. The term \textit{transverse-magnetic} (TM, or equivalently, \textit{p-polarized}) wave has its established meaning for oblique incidence on a homogeneous interface: it means that the magnetic field is perpendicular to the plane of incidence, and thus always parallel to the interface. The term \textit{transverse-electric} (TE, or, \textit{s-polarized}) denotes the opposite situation. 

Unfortunately, the same notation is used by many papers also for a perpendicular incidence on a grating or other structure with 1-D periodicity. In majority of cases "TM" denotes that \textit{the magnetic field is transverse to the translation axis of the structure, the electric field parallel to it} \cite{joannopoulos2011photonic, rybin2014photonic}. Alas, in numerous other cases the same term "TM" denotes an opposite interpretation, i.e., that \textit{the magnetic field is perpendicular to the 2-D plane that represents the whole electromagnetic problem, and the electric field component lies in this plane} \cite{vynck2009all}. 
Even greater complication arises under oblique illumination of a grating. These two different meanings of apparently identical terms can either come into a confusing discrepancy (TE+TM), an even more confusing agreement (TE+TE or TM+TM), or may also become inapplicable under general geometry of the wave or its polarisation.
%% TODO definition between TM (or, "p") and TE (or, "s") conventions in quasi-two-dimensional structures
%% ---- TM (or, "p") means: "electric field is along the translation axis of the structure, magnetic field along the grating vector"
%% * \cite{rybin2014photonic} "modes with the magnetic component in the xy plane, i.e., TM (Hx, Hy, Ez)"
%% * \cite{joannopoulos2011photonic} "(Hx, Hy, Ez) ... The latter, in which the magnetic field is confined to the xy plane, are called transverse-magnetic (TM) modes."
%% ---- TM (or, "p") means: "magnetic field is along the translation axis of the structure, electric field along the grating vector"
%% * \cite{vynck2009all} "dielectric rods in s-polarized light (electric field parallel to the axis of the rods) could exhibit both ..."
%% * Thibault's computations 
%% TODO in non-perpendicular incidence on a plane: 
%% ---- TM (or, "p") means "electric field parallel to the plane of incidence, magnetic field to the optical surface"
%% * \cite{jelinek2010fishnet} "TM incident wave with field components (Ey, Ez, Hx)
%% * everybody knows that "s"-polarisation reflects more than "p" at dielectric interfaces

\begin{table}[ht]   \caption{Table of abbreviations}  \label{tb_shortcuts} \centering 
\begin{tabular}{ll}
 \toprule
Abbreviation & Meaning	\\
 \hline
1-D, 2-D, 3-D & One-, two- and three-dimensional \\
CDH			& Current-Driven Homogenisation\\
EDB			& (Formulation of Maxwell equations using the $\E$, $\D$ and $\B$ vectors)\\
FDFD 		& Finite-difference frequency-domain (algorithm)\\
FDM 		& Filter diagonalisation method (algorithm)\\
FDTD 		& Finite-difference time-domain (algorithm)\\
FEM 		& Finite-element method (algorithm)\\
FFT 		& Fast Fourier transform (algorithm)\\
FRoI 		& Frequency range of interest\\
GVD 		& Group velocity dispersion \\
IFC			& Isofrequency contours\\
LHM			& Left-Handed (Meta)Material\\ 
MM			& Metamaterial\\
NGV 		& Negative group velocity\\
NRI 		& Negative refractive index\\
NRW 		& Nicolson-Ross-Weir (algorithm)\\
PBG			& Photonic band-gap\\
PhC 		& Photonic crystal\\
%PML 		& Perfectly matched layers (absorber in simulation)\\
PWEM 		& Plane-wave expansion method (algorithm)\\
RHM 		& Right-Handed (Meta)Material\\ 
SPP			& Surface Plasmon-Polariton\\		
SRR			& Split-ring resonator\\		
sSRR		& Symmetric split-ring resonator\\
STO			& Strontium titanate, SrTiO$_3$ (ferroelectric material)\\
TDTS 		& Time-domain terahertz spectroscopy\\
%TMM		& Transfer-matrix method  (numerical computation)\\
 \bottomrule
 \end{tabular} \end{table}

\begin{table}[ht]   \caption{Symbols used, approximately in the order they are introduced in text}  \label{tb_symbols} \centering 
\begin{tabular}{ll}
 \toprule
Symbol & Meaning	\\
 \hline
$\E$ 		& Electric field\\
$\E_0$ 		& Amplitude of the electric field\\
$\D$ 		& Electric displacement\\
$\HH$ 		& Magnetic field\\
$\B$ 		& Magnetic displacement									\vspace{3mm}\\
 
$\varepsilon_0$ &Vacuum permittivity, $8.85\cdot10^{-12}$ F/m\\
$\mu_0$		&Vacuum permeability, $1.25\cdot10^{-6}$ H/m			\vspace{3mm}\\
 
$\ii$		& Imaginary unit, $\ii^2 = -1$\\
$e$ 		& Euler constant, $e = 2.718\ldots$\\
$\pi$ 		& $\pi = 3.141\ldots$\\
$f$			& Frequency\\
$\omega$ 	& Angular frequency, $\omega = 2\pi f$\\
$\kk$ 		& Wave vector in homogeneous media\\
$\KK$ 		& Wave vector of the Bloch wave in periodic media\\
$t$, $\tau$ 		& Time\\
$c$ 		& Speed of light in vacuum, $c=2.998\cdot 10^8$ m/s \\
$f(t), F(\omega)$ & A function in the time domain, and in the frequency domain \\
$\chi_e,\chi_m$	& Electric and magnetic susceptibility in the local approximantion \\
$\epsrl(\omega), \murl(\omega)$ &Relative permittivity and permeability in the local approximation\\
$\epsrn(\omega, \kk)$ &Relative permittivity for nonlocal media\\
$\murn(\omega, \kk)$ &Relative permeability for nonlocal media\\
$\epsLL(\omega, \kk)$ &Relative permittivity in the Landau-Lifshitz ($\E\D\B$) formulation\\
$\Neff$ 	& Effective index of refraction (of periodic media)\\
$\Zeff$ 	& Effective impedance\\
$\eeff, \meff$ 	& Effective permittivity and permeability (of periodic media)\\
$\rr$, $\brho$ 		& Position in space (radius vector)\\
$\mathbf{a}_{1,2,3}$, $a$ 		& Lattice vectors, unit cell size in the cubic lattice \\
$\mathbb{R}$		& Real numbers\\
$\mathbb{Z}$		& Integers\\
$\mathbb{C}$		& Complex numbers\\
$h$ 		& Planck constant, $h = 6.626\cdot 10^{-34}$ J s\\

 \bottomrule
 \end{tabular} \end{table}
 %with a density $\rho$ and average velocity $\mathbf{v}(t)$
%}}}


 % TODO note about the field polarisation convention in most plots in the Results section
