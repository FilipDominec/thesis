\begin{center}
	\Large{\textbf{
Czech Technical University in Prague\\
Faculty of Nuclear Sciences and Physical Engineering\\ 
%Department of Physical Electronics\\
}}
 \vspace{3cm}

\includegraphics[width=5cm]{img/LogoCVUT}
\vspace{2cm}

\huge{\textbf{\scshape Dissertation thesis}}\\
\vspace{5mm}
{\LARGE\textbf{Metamaterials for the terahertz spectral range\\}}
\end{center}

\date{ } 			


 \vfill
\begin{minipage}{\textwidth}
\Large{\textbf{Prague 2015}} \hfill \Large{\textbf{Filip Dominec}}
\end{minipage}


\thispagestyle{empty} \newpage \setcounter{page}{1}

% --------------------------------------------------------------------------------
\chapter*{Bibliografický záznam}
 %\begin{tabular}{rl}
 %Author: 	&\textbf{Filip Dominec}\\
 %Advisor: 	&\textbf{Mgr. Filip Kadlec, Dr.}\\
 %Consultant: 	&\textbf{Doc. Ing. Ivan Richter, Dr.}\\ 
 %Year:		&\textbf{2015}\\
 %\end{tabular}

\bgroup \def\arraystretch{1.5}
\noindent\begin{tabular}{p{.25\linewidth}p{.7\linewidth}}
Autor:					&\textbf{Ing. Filip Dominec} \\
					~	&Czech Technical University in Prague\\
					~	&Faculty of Nuclear Sciences and Physical Engineering\\ 
					~	&Department of Physical Electronics\\
Název práce:			&\textbf{Metamateriály pro terahertzovou spektrální oblast} \\
Studijní program:		&\textbf{Aplikace přírodních věd} \\
Studijní obor:			&\textbf{Fyzikální inženýrství} \\
Školitel:				&\textbf{Mgr. Filip Kadlec, Dr.} \\
Školitel specialista:	&\textbf{Doc. Ing. Ivan Richter, Dr.} \\
Akademický rok:			&\textbf{2014/15} \\			%% TODO is this the correct format?
Počet stran:			&\textbf{\pageref{enddocument}} \\  %% TODO shall give the overall page number or just that of the authored text?
Klíčová slova			&\textbf{metamateriály, fotonické krystaly, terahertzová technologie, elektrodynamické simulace, homogenizace efektivních prostředí} \\
\end{tabular}
\egroup

\thispagestyle{empty} \newpage

% --------------------------------------------------------------------------------
\chapter*{Bibliographic entry}
\bgroup \def\arraystretch{1.5}
\noindent\begin{tabular}{p{.25\linewidth}p{.7\linewidth}}
Author:					&\textbf{Ing. Filip Dominec} \\
					~	&Czech Technical University in Prague\\
					~	&Faculty of Nuclear Sciences and Physical Engineering\\ 
					~	&Department of Physical Electronics\\
Title of Dissertation:	&\textbf{Metamaterials for the terahertz spectral range} \\
Degree Programme:		&\textbf{Application of natural sciences} \\
Field of Study:			&\textbf{Physical engineering} \\
Supervisor:				&\textbf{Mgr. Filip Kadlec, Dr.} \\
Supervisor specialist:	&\textbf{Doc. Ing. Ivan Richter, Dr.} \\
Academic Year:			&\textbf{2014/15} \\
Number of Pages:		&\textbf{\pageref{enddocument}} \\
Keywords				&\textbf{metamaterials, photonic crystals, terahertz technology, computational electrodynamics, homogenization of effective media} \\
\end{tabular}
\egroup
\thispagestyle{empty} \newpage

% --------------------------------------------------------------------------------
%\vspace{30mm}
%\textbf{\huge{Abstrakt} }
%\vspace{-5mm}

%\begingroup \renewcommand{\newpage}{}
\chapter*{Abstrakt}
\noindent ~
\todo{+ Z abstraktu musí být zřejmé, co je cílem předložené práce a jakých vědeckých výsledků doktorand osobně dosáhl.}

\vspace{10mm}
%\chapter*{Abstract} 
{\let\clearpage\relax\chapter*{Abstract}}
\noindent
~
\todo{+} 

%\endgroup

\thispagestyle{empty} \newpage
% --------------------------------------------------------------------------------
\chapter*{Acknowledgements}
%% Preparation of this thesis was not prepared without difficulties. However, it appears that overcoming such difficulties is necessary to gain some sort of valuable knowledge and experience that can not be conveyed through textbooks.
%%
%%
%%
%%
%%

The numerical results presented could be hardly obtained without the hard work of hundreds of volunteers contributing to the open-source scientific software: Most of the plots were made using the Matplotlib library \cite{hunter2007}, FDTD computations were based on MEEP \cite{oskooi2010meep} and PWEM computations used the MBP programs \cite{johnson2001mpb}. Harmonic inversion was used based on the FDM algorithm \cite{mandelshtam1997harmonic}.

This work was supported by the Czech Science Foundation under Grant No. 14-25639S.


\thispagestyle{empty} \newpage
