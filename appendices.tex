\section{Appendix A: from Fresnel reflection to $s$-parameter inversion} \label{app_fresnel} % TODO
%\subsection{Frequency-band source for oblique incidence in FDTD} % TODO add this?
\subsection{Fresnel equations} % TODO
\mdf{
also known as Fresnel-Airy function; 
the resulting Fabry-Pérot resonances result in the typical Fresnel-Airy function in the transmittance spectrum;
(note that the term of \textit{Airy function} is used also for completely different functions thorough physics, such as the eigenstate of a particle in sawtooth potential well, or the diffraction pattern from a straigth edge)
}
\subsection{Transmission of a slab} % TODO
\subsection{Derivation of inverse }\label{app_fresnel_inv} % TODO
%% TODO rewrite my "cai-shalaev" notes here


\section{Appendix C: Scripts used for numerical simulations} 
% TODO
% Permittivity spectra of selected materials can be found on % TODO

\section{Appendix D: Notation} % TODO
% Preliminary


%% we will try to build the theory from first principles


\paragraph{Abbreviations used, ordered alphabetically} %{{{
\begin{table}[ht]   \caption{Table of abbreviations}  \label{tb_shortcuts} \centering 
\begin{tabular}{ll}
 \toprule
Abbreviation & Meaning	\\
 \hline
EDB		& (Formulation of Maxwell equations using the $\E$, $\D$ and $\B$ vectors)\\
FDTD 		& Finite-difference time-domain (algorithm)\\
FDFD 		& Finite-difference frequency-domain (algorithm)\\
FEM 		& Finite-element method (algorithm)\\
FFT 		& Fast Fourier transform (algorithm)\\
FRoI 		& Frequency range of interest\\
FDM 		& Filter diagonalisation method (algorithm)\\
GVD 		& Group velocity dispersion \\
IFC		& Isofrequency contours\\
LHM		& Left-Handed Material\\ %% TODO used anywhere?
MM		& Metamaterial\\
NRW 		& Nicolson-Ross-Weir (method for effective parameter retrieval)\\
NGV 		& Negative group velocity\\
PBG		& Photonic band-gap\\
PhC 		& Photonic crystal\\
PML 		& Perfectly matched layers (absorber in simulation)\\
PWEM 		& Plane-wave expansion method (algorithm)\\
RHM 		& Right-Handed Material\\ %% TODO used anywhere?
STO		& Strontium titanate, SrTiO$_3$ (ferroelectric material)\\
SRR		& Split-ring resonator\\
TDTS 		& Time-domain terahertz spectroscopy\\
%TMM			& Transfer-matrix method  (numerical computation)\\
 \bottomrule
 \end{tabular} \end{table}
%}}}

\begin{table}[ht]   \caption{Symbols used, approximately in the order they are introduced in text}  \label{tb_symbols} \centering %{{{
\begin{tabular}{ll}
 \toprule
Symbol & Meaning	\\
 \hline
$\E$ 		& Electric field\\
$\E_0$ 		& Amplitude of the electric field\\
$\D$ 		& Electric displacement\\
$\HH$ 		& Magnetic field\\
$\B$ 		& Magnetic displacement									\vspace{3mm}\\
 
$\varepsilon_0$ &Vacuum permittivity, $8.85\cdot10^{-12}$ F/m\\
$\mu_0$		&Vacuum permeability, $1.25\cdot10^{-6}$ H/m			\vspace{3mm}\\
 
$\ii$		& Imaginary unit, $\ii^2 = -1$\\
$e$ 		& Euler constant, $e = 2.718\ldots$\\
$\pi$ 		& $\pi = 3.141\ldots$\\
$f$			& Frequency\\
$\omega$ 	& Angular frequency, $\omega = 2\pi f$\\
$\kk$ 		& Wave vector in homogeneous media\\
$\KK$ 		& Wave vector of the Bloch wave in periodic media\\
$t$, $\tau$ 		& Time\\
$c$ 		& Speed of light in vacuum, $c=2.998\cdot 10^8$ m/s \\
$f(t), F(\omega)$ & General function in time domain, and its counterpart in Frequency domain \\
$\chi_e(t),\chi_m(t)$	& Electric and magnetic susceptibility in the local approximantion \\ %% TODO separate freq-domain and time-domain susc?
$\chi_e(\omega)$, $\chi_m(\omega)$ 	& Electric and magnetic susceptibility\\
$\epsrl(\omega), \murl(\omega)$ &Relative permittivity and permeability in the local approximation\\
$\epsrn(\omega, \kk)$ &Relative permittivity for nonlocal media\\
$\murn(\omega, \kk)$ &Relative permeability for nonlocal media\\
$\epsLL(\omega, \kk)$ &Relative permittivity in the Landau-Lifshitz ($\E\D\B$) formulation\\
$\Neff$ 	& Effective index of refraction (of periodic media)\\
$\Zeff$ 	& Effective impedance\\
$\eeff, \meff$ 	& Effective permittivity and permeability (of periodic media)\\
$\rr$, $\brho$ 		& Position in space (radius vector)\\
$\mathbf{a}_{1,2,3}$, $a$ 		& Lattice vectors, unit cell size in the cubic lattice \\
$\mathbb{R}$		& Real numbers\\
$\mathbb{Z}$		& Integers\\
$\mathbb{C}$		& Complex numbers\\
$h$ 		& Planck constant, $h = 6.626\cdot 10^{-34}$ J s\\

 \bottomrule
 \end{tabular} \end{table}

\mdf{General notation of vectors and scalars} 
%}}}

