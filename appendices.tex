\section{Appendix A: from Fresnel reflection to $s$-parameter inversion} \label{app_fresnel} % TODO
%\subsection{Frequency-band source for oblique incidence in FDTD} % TODO add this?
\subsection{Fresnel equations} % TODO
\mdf{
also known as Fresnel-Airy function; 
the resulting Fabry-Pérot resonances result in the typical Fresnel-Airy function in the transmittance spectrum;
(note that the term of \textit{Airy function} is used also for completely different functions thorough physics, such as the eigenstate of a particle in sawtooth potential well, or the diffraction pattern from a straigth edge)
}
\subsection{Transmission of a slab} % TODO
\subsection{Derivation of inverse }\label{app_fresnel_inv} % TODO
%% TODO rewrite my "cai-shalaev" notes here

\section{Appendix B: Permittivity spectra of selected materials} % TODO

\section{Appendix C: Scripts used for numerical simulations} % TODO

\section{Appendix D: Notation} % TODO
% Preliminary


%% we will try to build the theory from first principles

\paragraph{General notation of vectors and scalars} %{{{
%}}}


\paragraph{Abbreviations used} %{{{
\begin{table}[ht]   \caption{Table of abbreviations}  \label{tb_shortcuts} \centering 
\begin{tabular}{ll}
 \toprule
Abbreviation & Meaning	\\
 \hline
FDTD 		& Finite-difference time-domain (numerical simulation)\\
FDFD 		& Finite-difference frequency-domain (numerical simulation)\\
FEM 		& Finite-element method (numerical simulation)\\
FFT 		& Fast Fourier transform (algorithm)\\
FRoI 		& Frequency range of interest\\
FDM 		& Filter diagonalisation method (algorithm)\\
IFC			& Isofrequency contours\\
MM			& Metamaterial\\
NRW 		& Nicolson-Ross-Weir (method for effective parameter retrieval)\\
PBG			& Photonic band-gap\\
PhC 		& Photonic crystal\\
PML 		& Perfectly matched layers (absorber in simulation)\\
PWEM 		& Plane-wave expansion method (numerical simulation)\\
STO			& Strontium titanate, SrTiO$_3$ (ferroelectric material)\\
TDTS 		& Time-domain terahertz spectroscopy\\
 \bottomrule
 \end{tabular} \end{table}
%}}}

\begin{table}[ht]   \caption{Symbols used}  \label{tb_symbols} \centering %{{{
\begin{tabular}{ll}
 \toprule
Symbol & Meaning	\\
 \hline
$\E$ 		& Electric field\\
$\E_0$ 		& Amplitude of the electric field\\
$\D$ 		& Electric displacement\\
$\HH$ 		& Magnetic field\\
$\B$ 		& Magnetic displacement\\
$f(t), F(\omega)$ & General function in time domain, and its counterpart in Frequency domain \\
$\chi_e(t),\chi_m(t)$	& Electric and magnetic susceptibility in the local approximantion \\ %% TODO separate freq-domain and time-domain susc?
$\chi_e(\omega)$, $\chi_m(\omega)$ 	& Electric and magnetic susceptibility\\
$\varepsilon_0$ &Vacuum permittivity, $8.85\cdot10^{-12}$ F/m\\
$\mu_0$		&Vacuum permeability, $1.25\cdot10^{-6}$ H/m \\
$\epsrl(\omega)$ &Relative permittivity in the local approximation\\
$\murl(\omega)$ &Relative permeability in the local approximation\\
$\epsrn(\omega, \kk)$ &Relative permittivity for nonlocal media\\
$\murn(\omega, \kk)$ &Relative permeability for nonlocal media\\
$\epsLL(\omega, \kk)$ &Relative permittivity in the Landau-Lifshitz ($\E\D\B$) formulation\\
%% TODO add eps, epsLL, mu_r, mu_LL, and distinguish between the symmetric and Landau-Lifshitz form
$\Neff$ 	& Effective index of refraction (of periodic medium)\\
$\eeff$ 	& Effective permittivity\\
$\meff$ 	& Effective permeability\\
$\Zeff$ 	& Effective impedance\\
$\rr$, $\brho$ 		& Position in space (radius vector)\\
$\mathbf{a}_{1,2,3}$, $a$ 		& Lattice vectors, unit cell size in the cubic lattice \\
$t$, $\tau$ 		& Time\\
$\omega$ 	& Angular frequency\\
$\kk$ 		& Wave vector in homogeneous medium\\
$\KK$ 		& Wave vector of the Bloch wave in periodic medium\\
$\mathbb{R}$		& Real numbers\\
$\mathbb{Z}$		& Integers\\
$\mathbb{C}$		& Complex numbers\\
$\ii$		& Imaginary unit, $\ii^2 = -1$\\
$e$ 		& Euler constant, $e = 2.718\ldots$\\
$h$ 		& Planck constant, $h = 6.626\cdot 10^{-34}$ J s\\
$\mathcal F$ 		& Fourier transform\\

 \bottomrule
 \end{tabular} \end{table}
%}}}

