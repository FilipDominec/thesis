
%% CONCLUSION =
%% * overall analysis and integration of the research and conclusions of the thesis in
%%   light of current research in the field
%% * conclusions regarding goals or hypotheses of the thesis that were presented in
%%   the Introduction, and the overall significance and contribution of the thesis
%%   research
%% * comments on strengths and limitations of the thesis research
%% * discussion of any potential applications of the research findings
%% * an analysis of possible future research directions in the field drawing on the work
%%   of the thesis
%     --> 

%% NOTES: What was not discussed in the thesis
%% quasicrystals, chiral media and gyrotropy, structures with nonlinearity or loss,    tunable structures

%%% TODO TODO TODO TODO TODO TODO 
%%% Hole : Conclusion - what to put here?
%%% TODO TODO TODO TODO TODO TODO 

\add{
\section{Metamaterials -- Critique and alternatives}
Many different designs of photonic crystals, and in particular, of metamaterials were proposed in the last decades, part of them being aimed at the terahertz range. %They have been also summed by several books and reviews % TODO refs x5
lacks:

TODO TODO At the end, the author would like to express his concern about the efficiency and practical impact of the metamaterial research in the last years. 
The number of papers related to metamaterials has rocketed in the last decade in almost unprecedented way % todo cite some MM scientometry here

, and 
the hypertrophic number of publications dissolves the concentration of pertinent ideas in almost a homeopatic way.  
Perhaps anybody interested in MM will soon notice there are many papers adding just a quantitative difference to what was already published, often without stating this important fact. This may give a superficial feeling of an exponential gathering of knowledge, but in fact it mostly contributes to the information noise. 
In addition, the MM-related books available are often composed of ordinary papers, unrelated to each other and not covering all necessary theory, thus hardly preparing the reader to study the literature.

%The name of this chapter is borrowed from the same-named book by Ben Munk \cite{Munk2004}. The book presents several objections against MM that seem to be ill-argumented and some of them seem to be also utterly wrong. 
%However, as another point it also touches a really deep problem: 

It is understandable that such a new (or at least newly recognized) scientific field exerts a great pressure to publish as much and as early as possible, which is exacerbated by the objective way of how the contemporary scientific work is evaluated. 
That said, the situation is not as bad as it might seem and the author is also aware of several recent brilliant MM publications that give a deep insight into new physical phenomena.


Another question is the practical benefit of metamaterials for the humankind. 
Compared to the publication boom and its optimistic aims, the metamaterials find their way to applications markedly slowly. The novel phenomena such as cloaking or sub-wavelength imaging are still confined to laboratory demonstrations, usually for structures not much bigger than the wavelength. The research of semiconductors in the first decades did not bring any notable applications, either, but even at that time it had to be clear that semiconductors present a wholly new field of basic material research. On the other hand, the recent progress in the field of metamaterials resembles the applied research: It is mostly a \textit{design} and \textit{optimisation} of already known structures towards some particular behaviour, rather than a search for new fundamental physics. 

Writing these lines, the author is not aware of any readily realized application where a metamaterial would present a significant advantage over classical structure. Will metamaterials ever find their application in any field of technology? And if not, will at least some of the developed theory or methods be useful in the future?

To conclude this pessimistic, but well meant section, the author
}

%% First, reduction of losses by using crystalline metals and/or by introducing optically amplifying materials; developing three-dimensional isotropic designs rather than planar structures; and finding ways of mass producing large-area structures. -- Soukolis in http://eurekalert.org/pub_releases/2007-01/dl-mft010407.php?light


%% """ If a too-simplistic physical theory is forced to predict or explain data that are not within its scope,
%%     it can often do so, but without giving any physical insight. Such a result is not a proof of the theory to be
%%     applicable to the problem, and in some cases not even of the theory to make any sense whatsoever."""

Fundamental issues have to be resolved in the MM and PhC field are
losses

The TiO$_{2}$ resonators, whose the manufacturing and sieving process, experimental characterisation and numerical computation was extensively discussed in this thesis, can be viewed as an example of such research.
\add{Owing to the homogeneous broadening of the resonator resonance, caused by intrinsic losses of TiO$_{2}$, further sieving appears to lack any purpose. Not only this project broadens the numerous group of technically successful research projects pointed in utterly wrong direction, it also classifies into the lamentable subgroup of effort that could be entirely avoided if open and honest scientific discussion took place about the inherent problems. In this case, particularly, in all related papers it should be openly stated that the dielectric losses of TiO2 prevent building any applicable volume metamaterial, no matter how accurately the resonators are prepared.}

\paragraph{Optimistic notes}
That said, the author does not want the thesis to sound too sceptical about the future of electromagnetic structures. 

\add{
Metamaterials do not bring new physical phenomena on their own, they only force one to conscientiously review the common electrodynamics. There
are no novel phenomena, just a novel task of homogenisation and possibly enew values of constitutive parameters"

The research of Metamaterials is typical by taking a relatively novel phenomenon (wave propagation in periodic structure) and
interpreting it on the basis of a well-developed theory (electrodynamics of homogeneous media). In future, this trend may be opposite --
that is, new theory motivated by the metamaterial research may find its application on other phenomena.
One example may be the theory of strongly hyperbolic media, which turns out to be fundamental for electrodynamics in extremely strong magnetic fields \cite{smolyaninov2011vacuum}.

}

\section{Outcomes of the thesis}



We presented experimental and theoretical studies of simple periodic structures -- the periodic arrays of dielectric slabs, two orientations of rods, dielectric spheres, metallic wire mesh, as well as a composite of spheres and a mesh. An attempt was made to make an analysis of the modes that are responsible for the photonic bands, illustrating different electromagnetic phenomena and their dependence on a structure parameter. An analytic model for wire medium was proven to match our numerical results.
The prospects for the future work are to broaden this catalogue to other structures in order to enable comparison and deeper physical interpretation of their behaviour. 

A versatile numerical environment for simulations was prepared and tested in the last two years. All computation scripts, containing over 2000 lines of Python code, are open-source and they use freely-redistributable libraries only, which may be useful for the broader scientific community in the future. 
The automated effective parameters retrieval from FDTD is especially suitable for parametric scans. It proves instrumental in optimisation of the structure performance, matching experimental data, as well as studying the tunability.

Broad range of physical phenomena may be included in the simulations. A library of realistic permittivity models is being built that allows to match TDTS experiments with a good accuracy. It is also possible to combine dispersive dielectrics with metals and doped semiconductors. 
 The FDTD method is applicable also to non-periodic, lossy and time-changing structures, while PWEM enables e. g. studies of the magnetooptic phenomena.
All referred numerical methods, FDTD, PWEM and TMM, were tested also to give compatible and plausible results under oblique incidence and both TE/TM polarisations. 

%%26. Y. Fu, L. Thyl ́n and H. Agren, “A Lossless Negative Dielectric Constant from Quantum e Dot Exciton Polaritons,” Nano Lett. 8, 1551 (2008).
%%27. A. Bratkovsky, E. Ponizovskaya, S. Y Wang, P. Holmstr ̈m, L. Thyl ́n, Y. Fu, and o e H.  ̊gren, “A Metal-wire/Quantum-dot Composite Metamaterial with Negative ǫ and A Compensated Optical Loss,” Appl. Phys. Lett. 93, 193106 (2008).
%%28. Y. Zeng, Q. Wu and D. H. Werner, “Electrostatic Theory for Designing Lossless Negative Permittivity Metamaterials,” Opt. Lett. 35, 1431 (2010).

This work was supported by the Czech Science Foundation under Grant No. 14-25639S.

\mdf{Most of the plots were made using the Matplotlib library \cite{hunter2007}, FDTD computations were based on MEEP \cite{oskooi2010meep} and PWEM computations used the MBP program \cite{johnson2001mpb}. Harmonic inversion was used based on the FDM algorithm \cite{mandelshtam1997harmonic}}
